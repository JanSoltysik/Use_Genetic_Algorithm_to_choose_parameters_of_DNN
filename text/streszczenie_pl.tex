\documentclass{article}

\usepackage{polski}
\usepackage[utf8]{inputenc}

\title{Zastosowanie algorytmu genetycznego do optymalizacji struktury sieci neuronowej - Streczenie}
\author{Jan Sołtysik}

\begin{document}
\maketitle
Celem pracy było zaproponowanie i zaimplementowanie algorytmu genetycznego (AG) umożliwiającego znalezienie optymalnej struktury sztucznej sieci neuronowej (SSN) zarówno dla problemu klasyfikacji jak i regresji. Budowanie modeli SSN jest procesem skomplikowanym, współczesne biblioteki implementujące SSN znacznie upraszczają ich używanie, lecz wciąż to użytkownik musi wybrać multum hiperparametrów sieci takich jak: ilość i rodzaje warstw, funkcje aktywacyjne itd. Aby zautomatyzować proces wyboru hiperparametrów zaproponowane zostały modyfikacje prostego algorytmu genetycznego – sposób kodowania sieci neuronowej oraz operatory genetyczne. Do zrealizowania powyższego celu wykorzystaany został język programowania \texttt{python} wraz z popularnymi bibliotekami  \texttt{numpy}  oraz  \texttt{tensorflow}.  Zaimplementowany w ramach pracy algorytm został przetestowany na dwóch zbiorach danych: \textbf{Fashion-Mnist} - zadanie rozpoznania kategorii ubrania na podstawie jego zdjęcia, oraz \textbf{Boston Housing} - przewidywanie ceny domu w zależności od wartości zadanych dla jego parametrów.
\end{document}
